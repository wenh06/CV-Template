\ifPhdThesisSection

\ifChinese

\section{博士论文}

\cvitem{题目}{\bfseries 定义在数域上的超曲面的重数计数问题}
\cvitem{导师}{{\bfseries 张贺春} 教授}
\cvitem{评审人}{唐舜,徐飞(首都师范大学),\newline
王崧(中国科学院晨兴数学中心),\newline
姚家燕(清华大学)}
\cvitem{答辩委员会}{徐飞({\bfseries 主席},首都师范大学),\newline
郑维喆(中国科学院晨兴数学中心),\newline
姚家燕,朱斌,邓邦明(清华大学)}
\cvitem{论文简介}{考虑定义在数域上的射影超曲面,可以通过局部 Hilbert-Samuel 函数定义其上点的重数。取定一个计数函数,考虑在这个射影超曲面上代数点的重数的计数问题:即对超曲面上,高度不超过定值,定义域相对于基域的扩张次数等于一个定值的所有代数点,用取定的计数函数作用在这些点的重数上,求和。论文对这个问题给出了一个上界估计,这个上界将会与射影超曲面的次数,奇点集的维数,高度的上界,以及相应域扩张次数有关。这个估计可以在某种程度上衡量射影超曲面的奇点集的复杂程度。}

\else  % else of \ifChinese

\section{Ph.D. Thesis}

\cvitem{Title}{\emph{Counting Multiplicities in a Hypersurface over Number Fields}}
\cvitem{Supervisor}{Professor {\bfseries Hechun ZHANG}}
\cvitem{Reviewers}{Shun TANG, Fei XU (Capital Normal University),\newline Song WANG (Chinese Academy of Sciences),\newline Jiayan YAO (Tsinghua University)}
\cvitem{Committee Members}{Fei XU ({\bfseries Chair}, Capital Normal University),\newline Weizhe ZHENG (Chinese Academy of Sciences),\newline Jiayan YAO, Bin ZHU, Bangming DENG (Tsinghua University)}
\cvitem{Description}{Fix a counting function of multiplicities of algebraic points in a projective hypersurface over a number field, and take the sum over all algebraic points of bounded height and fixed degree. An upper bound for the sum with respect to this counting function is given in terms of the degree of the hypersurface, the dimension of the singular locus, the upper bounds of height, and the degree of the field of definition. This upper bound describes the complexity of this hypersurface's singular locus to some extent.}

\fi  % fi of \ifChinese

\fi  % fi of \ifPhdThesisSection
