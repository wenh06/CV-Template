
% \usepackage[style=authoryear,sorting=ydnt,dashed=false,maxbibnames=99]{biblatex}

% \renewbibmacro*{date}{}
% \renewbibmacro*{date+extrayear}{}
% \renewbibmacro*{issue+date}{}
% \newcommand*{\bibyear}{}

% \defbibenvironment{bibliography}
% {\list
%  {\iffieldequals{year}{\bibyear}
%     {}
%     {\printfield{year}%
%      \savefield{year}{\bibyear}}}
%  {\setlength{\topsep}{0pt}% layout parameters based on moderncvstyleclassic.sty
%   \setlength{\labelwidth}{\hintscolumnwidth}%
%   \setlength{\labelsep}{\separatorcolumnwidth}%
%   \setlength{\itemsep}{\bibitemsep}%
%   \leftmargin\labelwidth%
%   \advance\leftmargin\labelsep}%
%   \sloppy\clubpenalty4000\widowpenalty4000}
% {\endlist}
% {\item}

% \addbibresource{publications.bib}

\usepackage{moderntimeline}
\tlmaxdates{2016}{2023}

\newif\ifnumericCVbibliography
\numericCVbibliographytrue % replace true with false to disable
\newlength{\hintscolumnwidthV}
\setlength\hintscolumnwidthV{\hintscolumnwidth}
\newlength{\labelkern}
\ifnumericCVbibliography
  \setlength\labelkern{-2ex}
  \usepackage[backend = biber,
              style   = nature,
              sorting=ydnt
             ]{biblatex}
  \newcommand{\printbibnumber}{%\hspace* here has no effect so I removed it
      \llap{\printtext[labelnumberwidth]{%
              \printfield{labelprefix}%
              \printfield{labelnumber}
              }\kern\labelkern% to reduce space between label and timeline image
            }%
  }
\else
  \usepackage[backend = biber,
              style   = authoryear,
              sorting=ydnt
             ]{biblatex}
  \newcommand{\printbibnumber}{\relax} % do nothing
  \setlength\labelkern{0ex}
  \setlength\hintscolumnwidthV{\hintscolumnwidth} % this length needs 
  % to be re-set or it keeps the numeric  version value if that has been run before
\fi
\addbibresource{publications.bib}
%-----------------------------------------------------------
\makeatletter
\newcommand*{\cventryV}[1][.25em]{}
% timeline for bibliography
\newcommand{\tldatecventryV}[2][color1]{%
\issincefalse
\tl@formatstartyear{#2}
\cventryV{\tikz[baseline=0pt]{
    \useasboundingbox (0ex,0ex) rectangle (\hintscolumnwidthV,1ex); 
    %changed origin of boundingbox. previous was (2ex,0ex) but this 
    % creates alignment problems when switching between numeric and non numeric.
    \fill [\tl@runningcolor] (0,0)
       rectangle (\hintscolumnwidthV,\tl@runningwidth);
    \fill [#1] (0,0)
       ++(\tl@startfraction*\hintscolumnwidthV,0pt)
       node [tl@singleyear] {#2}
       node {$\bullet$};
  }}}
\makeatother
%-----------------------------------
\defbibenvironment{bibliography}
  {\list
   {\printbibnumber% here you see the macro in action
    \tldatecventryV{%
      \thefield{year} % actual year from bibitem
       }}
       {%
        \setlength{\topsep}{0pt}% layout parameters based on moderncvstyleclassic.sty
        \addtolength\hintscolumnwidthV{-\labelnumberwidth}% num - changes timeline image length
        \addtolength\hintscolumnwidthV{\labelkern}% num  - changes timeline image length
        \setlength{\bibhang}{\hintscolumnwidthV} % custom bibhang
        \setlength{\labelsep}{\separatorcolumnwidth} %  horizontal distance between label and entry
        \setlength{\leftmargin}{\hintscolumnwidth} % sets where the left margin is
        \addtolength{\leftmargin}{\separatorcolumnwidth} %
        \setlength{\itemindent}{-\bibhang} % this sets indentation of the second line of the entry.
        % changing it moves also the first line left or right
        \addtolength{\itemindent}{-\separatorcolumnwidth} % to align the second line exactly
        \setlength{\itemsep}{\bibitemsep} % vertical distance between bib items
        \setlength{\parsep}{\bibparsep}}}
  {\endlist}
  {\item}
%-----------------------
\AtEveryBibitem{\clearfield{year}}
%-------------------------------------------------------------------------------
%  reverse numbering of publications
%-------------------------------------------------------------------------------
% Count total number of entries in each refsection
\AtDataInput{%
  \csnumgdef{entrycount:\therefsection}{%
    \csuse{entrycount:\therefsection}+1}}
% Print the labelnumber as the total number of entries in the
% current refsection, minus the actual labelnumber, plus one
\DeclareFieldFormat{labelnumber}{\mkbibdesc{#1}}
\newrobustcmd*{\mkbibdesc}[1]{%
  \number\numexpr\csuse{entrycount:\therefsection}+1-#1\relax}
